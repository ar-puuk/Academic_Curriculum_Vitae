%-----------PROJECTS-----------%
\section{\faIcon{tasks} \textbf{ Projects}}
\resumeSubHeadingListStart
    
    \resumeProjectHeading
    {\textbf{Greater Dalton 2050 Metropolitan Transportation Plan, GA} $|$ \emph{Metro Analytics}}{2024}
    \resumeItemListStart
        \resumeItem{For the Greater Dalton 2050 Metropolitan Transportation Plan, base year and future year socioeconomic data were prepared for travel demand model calibration. Existing data at the Traffic Analysis Zone (TAZ) level were compiled from the US Decennial Census, American Community Survey, and Longitudinal Employer-Household Dynamics data, then calibrated using satellite imagery and local sources. Projections were developed based on state control totals and best practices. The result was a set of socioeconomic data, including population, household, and employment characteristics, to be used in calibrating trip generation and travel demand at each TAZ.}
    \resumeItemListEnd

    \resumeProjectHeading
    {\textbf{Bowling Green – Warren County 2050 Metropolitan Transportation Plan, KY} $|$ \emph{Metro Analytics}}{2024}
    \resumeItemListStart
        \resumeItem{For the Bowling Green-Warren County 2050 Metropolitan Transportation Plan, the regional travel demand model was calibrated using existing traffic data and modified for future no-build, fiscally constrained, and unconstrained funding scenarios. Model outputs were summarized, maps of projected traffic volumes and congestion were generated, and technical memoranda with scenario analysis and statistics were produced to support decision-making in the MTP development.}
    \resumeItemListEnd
    
    \resumeProjectHeading
    {\textbf{\href{https://tmacog.org/transportation/freight-plan}{TMACOG Regional Freight Plan, OH}} $|$ \emph{Metro Analytics}}{2024}
    \resumeItemListStart
        \resumeItem{The Toledo Metropolitan Area Council of Governments required an updated freight profile to support future investments in transportation infrastructure. The challenge was to address safety, congestion, and environmental concerns while optimizing freight operations. Mr. Bhandari played a pivotal role in the project by collecting and analyzing transportation data, including asset conditions, traffic safety records, and daily and truck traffic volumes. Using GIS mapping, he visually represented complex data, facilitating stakeholder understanding of existing gaps and future needs. His analyses informed targeted recommendations for improving freight mobility while considering emissions reduction and the integration of freight operations into land use and environmental planning efforts.}
    \resumeItemListEnd

    \resumeProjectHeading
    {\textbf{Valdosta-Lowndes MPO 2050 Metropolitan Transportation Plan, GA} $|$ \emph{Metro Analytics}}{2024}
    \resumeItemListStart
        \resumeItem{For the Valdosta-Lowndes MPO 2050 Metropolitan Transportation Plan, support was provided as a planner and analyst for the Metropolitan Transportation planning process. Socioeconomic data at the Traffic Analysis Zone (TAZ) level for both base and future years were prepared, and assistance was provided in GIS mapping of existing transportation assets, needs assessment, and project identification. The result was the preparation of TAZ-level socioeconomic data, GIS maps for the existing conditions report, and contributions to the needs assessment and final documentation.}
    \resumeItemListEnd

    \resumeProjectHeading
    {\textbf{\href{https://apps.trb.org/cmsfeed/TRBNetProjectDisplay.asp?ProjectID=4957}{NCHRP 08-146: Integrating Resiliency into Transportation System Operations}} $|$ \emph{Metro Analytics}}{2024}
    \resumeItemListStart
        \resumeItem{U.S. transportation agencies face increasing challenges from environmental disruptions, necessitating robust resiliency planning. This research was focused on identifying operational gaps and offering actionable strategies to integrate resiliency into agency operations. Mr. Bhandari developed a comprehensive database of resiliency resources and an Operational Resiliency Online Toolkit, featuring a resource explorer and a self-assessment tool. His work provided agencies with insights into system vulnerabilities and enabled them to proactively enhance their operational resilience. This initiative improved the integration of resiliency considerations into planning and programming, aligning with land use and environmental sustainability goals.}
    \resumeItemListEnd
    
    \resumeProjectHeading
    {\textbf{\href{https://artsfreightplan.com/}{ARTS Regional Freight Plan, GA \& SC}} $|$ \emph{Metro Analytics}}{2024}
    \resumeItemListStart
        \resumeItem{Text}
    \resumeItemListEnd
    
    \resumeProjectHeading
    {\textbf{DARTS 2050 Metropolitan Transportation Plan Update, GA} $|$ \emph{Metro Analytics}}{2024}
    \resumeItemListStart
        \resumeItem{The Dougherty Area Regional Transportation Study sought to craft a comprehensive long-term transportation strategy addressing both present and future regional needs through 2050. The challenge was to align transportation priorities with fiscal realities while incorporating land use and environmental considerations such as emissions and equity impacts. Mr. Bhandari prepared detailed GIS maps, managed extensive datasets, and developed fiscally constrained and unconstrained work programs based on multi-source revenue projections. His work ensured that transportation investments supported regional safety, accessibility, and environmental sustainability, culminating in a cohesive plan seamlessly integrated into the MPO’s Transportation Improvement Program and addressing the region’s growth challenges.}
    \resumeItemListEnd
    
    \resumeProjectHeading
    {\textbf{Albany-DARTS Resiliency Plan, GA} $|$ \emph{Metro Analytics}}{2024}
    \resumeItemListStart
        \resumeItem{The Albany region faced significant vulnerabilities from frequent severe weather events, such as floods and hurricanes, that threatened transportation and community assets. Mr. Bhandari conducted comprehensive geospatial analyses, collecting and examining data on FEMA 100-year floodplains, storm-damaged areas, and historical disruptions.  By integrating land use and environmental considerations, he developed actionable recommendations to enhance equity, safety, and resilience within the transportation system. His work contributed to the region’s ability to mitigate future disruptions and improve overall preparedness.}
    \resumeItemListEnd
                
    \resumeProjectHeading
    {\textbf{Missouri DOT STIP 2024-2027 \& STIP 2025-2028, MO} $|$ \emph{Metro Analytics}}{2023 \& 2024}
    \resumeItemListStart
        \resumeItem{Missouri’s transportation network required strategic investments to preserve its aging infrastructure while addressing environmental and safety challenges. The project objective was to evaluate the societal benefits of these investments using robust analytical frameworks. Mr. Bhandari employed USDOT’s guidance, analyzing data from the Highway Performance Monitoring System and National Bridge Inventory to assess the impacts on safety, emissions, and infrastructure performance. His findings, detailed in technical memoranda, provided critical insights into the value of preservation projects, ensuring alignment with state and regional planning priorities and promoting sustainable infrastructure management.}
    \resumeItemListEnd
    
    \resumeProjectHeading
    {\textbf{Economic Impact Analysis of Utah Transit Authority, UT} $|$ \emph{Metro Analytics}}{2023-2024}
    \resumeItemListStart
        \resumeItem{The Utah Transit Authority required an analysis of transit investments’ economic benefits across diverse funding and operational scenarios. The challenge involved integrating various datasets, including those from the regional travel demand model (TDM), the regional Real Estate Market Model (REMM), and historical Land Information Records (LIR), to assess land value sensitivities and regional growth impacts. Mr. Bhandari leveraged his expertise in economic modeling and geospatial analysis to evaluate societal benefits such as improved market access, transportation efficiency, and emissions reductions. His findings underscored the role of transit in fostering sustainable land use and equitable access, providing a foundation for data-driven transit planning and investment decisions.}
    \resumeItemListEnd
    
    \resumeProjectHeading
    {\textbf{\href{https://www.stonecrestga.gov/MajorPlansAndStudies.aspx}{Stonecrest Freight Cluster Plan, GA}} $|$ \emph{Metro Analytics}}{2023-2024}
    \resumeItemListStart
        \resumeItem{In cooperation with the Atlanta Regional Commission (ARC), the City of Stonecrest sought a strategy to manage rising freight demands while maintaining community and environmental priorities. Mr. Bhandari conducted thorough analyses of transportation infrastructure and traffic data, worked with stakeholders to identify needs, and reviewed industry trends to project future freight growth. His work led to a balanced plan recommending short- and long-term projects that supported sustainable freight operations. These recommendations ensured that freight system demands were met without compromising community well-being, while aligning freight development with broader land use and environmental planning goals.}
    \resumeItemListEnd
    
    \resumeProjectHeading
    {\textbf{Economic Benefit and Impacts of Utah’s Unified Plan 2023-2050, UT} $|$ \emph{Metro Analytics}}{2023}
    \resumeItemListStart
        \resumeItem{The Unified Plan aimed to guide Utah’s transportation investments to maximize societal and economic benefits under varying funding scenarios. The challenge was to quantify the broader impacts of these investments, including accessibility, emissions, and safety outcomes, to inform decision-making. Mr. Bhandari utilized travel demand model outputs and the IMPLAN economic model to assess impacts on income growth, employment, and government fiscal gains. He meticulously evaluated how different investment scenarios influenced land use and regional connectivity, delivering a detailed report that provided stakeholders with actionable insights to prioritize funding for projects aligned with long-term environmental and economic objectives.}
    \resumeItemListEnd
    
    \resumeProjectHeading
    {\textbf{Ohio DOT Cleveland CCG3A Intersection, OH} $|$ \emph{Metro Analytics}}{2023}
    \resumeItemListStart
        \resumeItem {The Ohio DOT aimed to secure federal funding for the Cleveland Innerbelt Modernization Plan’s Center Interchange reconstruction to address long-term congestion and safety issues. The objective of the project was to demonstrate the project’s societal benefits, including emissions reductions and improved accessibility, to meet federal grant requirements. Mr. Bhandari conducted a detailed benefit-cost analysis using USDOT guidelines and travel demand model outputs, meticulously quantifying the project’s impacts. His analysis underscored the alignment of the proposed improvements with regional environmental and land use planning objectives, successfully supporting the project’s funding application.}
    \resumeItemListEnd

    \resumeProjectHeading
    {\textbf{\href{https://bhandaripukar.com.np/pt-portfolio/downtown-slc-parking-study/}{Downtown Salt Lake City Parking Study}} $|$ \emph{University of Utah (MCMP Professional Project)}}{2023}
    \resumeItemListStart
        \resumeItem{A parking study for Downtown Salt Lake City was conducted as part of a professional project for the Master’s in City \& Metropolitan Planning at the University of Utah, in collaboration with Wasatch Front Regional Council (WFRC). The project involved reviewing the utilization of off-street parking within the Free Fare Zone using satellite imagery, site visits, and temporal data collection at multiple times of the day. The analysis of spatial and temporal parking occupancy informed recommendations for parking demand management policies. The study provided actionable strategies for parking demand, supply, and management, and was recognized as the "Best Professional Project of the Year" by the Department of City \& Metropolitan Planning.}
    \resumeItemListEnd

    \resumeProjectHeading
    {\textbf{\href{https://bhandaripukar.com.np/pt-portfolio/travel-demand-modeling/}{Four-Step Travel Demand Modeling}} $|$ \emph{University of Utah}}{2023}
    \resumeItemListStart
        \resumeItem{A project on Four-Step Travel Demand Modeling was completed as part of the CVEEN 6560 - Transportation Planning course at the University of Utah. The objective was to understand the process of travel demand modeling using TransCAD and R. The project involved applying the travel demand model theory to a sample from the Utah Household Survey and other publicly available data, completing all four steps: generation, distribution, mode choice, and assignment. The result was a comprehensive documentation of the modeling process and analysis findings.}
    \resumeItemListEnd

    \resumeProjectHeading
    {\textbf{\href{https://bhandaripukar.com.np/pt-portfolio/salt-lake-city-surface-parking-occupancy-analysis/}{Salt Lake City Surface Parking Occupancy Analysis}} $|$ \emph{University of Utah}}{2023}
    \resumeItemListStart
        \resumeItem{This project served as the final assignment for the CVEEN 6115 - Data Science for Civil Engineering course at the University of Utah. The task involved employing machine learning algorithms to automate the counting of surface parking occupancy in Downtown Salt Lake City. Using Python and the scikit-learn library, a model was developed to detect vehicles from publicly available satellite imagery data for training and testing purposes. The resulting model utilized Support Vector Machines (SVM) and Principal Component Analysis (PCA) algorithms, achieving a parking occupancy counting accuracy of 75\%.}
    \resumeItemListEnd

    \resumeProjectHeading
    {\textbf{\href{https://bhandaripukar.com.np/pt-portfolio/traffic-impact-study-at-2100s-2100e/}{Traffic Impact Study at 2100S/2100E}} $|$ \emph{University of Utah}}{2023}
    \resumeItemListStart
        \resumeItem{For the Traffic Impact Study at the 2100S/2100E intersection in Salt Lake City, the assignment focused on analyzing the impact of a new mixed-use development on local traffic. I was tasked with using Synchro/SimTraffic to propose upgrades to the intersection. To achieve this, I collected traffic data from UDOT sources, applied trip generation methodologies, and utilized both stochastic and deterministic modeling to evaluate traffic patterns with and without the proposed development. The resulting report detailed existing traffic conditions, modeled future scenarios, and recommended intersection improvements to mitigate the development’s impact.}
    \resumeItemListEnd

    \resumeProjectHeading
    {\textbf{\href{https://bhandaripukar.com.np/pt-portfolio/pedestrian-and-bicycle-safety-analysis/}{Built Environment Determinants of Walking and Biking Safety}} $|$ \emph{University of Utah}}{2022}
    \resumeItemListStart
        \resumeItem{Collaborated with UDOT and local partners to analyze built environment factors affecting pedestrian and bicycle crash severity for a Transportation Data Analytics workshop. Conducted geospatial analysis in R on UDOT crash data (2017-2021) to examine the impact of variables like speed limits, transit stops, intersection safety, and bicycle level-of-stress (LOS) on crash outcomes. The study found pedestrian crashes were more risk-driven, while cyclist crashes were more exposure-driven, with speed being a common risk factor. The findings recommended integrating risk and exposure assessments into policy and emphasized the importance of traffic calming measures for enhancing safety.}
    \resumeItemListEnd

    \resumeProjectHeading
    {\textbf{\href{https://www.arcgis.com/apps/MapJournal/index.html?appid=5c94536e1840429bbd5f0e56118b9858}{Evaluation of Social Equity of Transit Accessibility in Salt Lake County, UT}} $|$ \emph{University of Utah}}{2022}
    \resumeItemListStart
        \resumeItem{As part of Advanced GIS coursework, conducted a geospatial analysis to evaluate the relationship between social equity and transit accessibility at the census tract level in Salt Lake County, UT. Measured transit accessibility using proximity to transit stations and analyzed social equity variables—such as minority populations, renter-occupied households, and households without vehicles—using demographic data. Applied spatial autoregressive models to explore the connection between transit access and equity. The study revealed that social equity variables are significantly higher for areas within a 20-minute walking distance of TRAX lines, benefiting residents in these transit-adjacent neighborhoods.}
    \resumeItemListEnd

    \resumeProjectHeading
    {\textbf{\href{https://bhandaripukar.com.np/pt-portfolio/thriving-in-place-at-liberty-wells/}{Thriving in Place: Salt Lake City, UT}} $|$ \emph{University of Utah}}{2022}
    \resumeItemListStart
        \resumeItem{Contributed to Salt Lake City's "\href{https://thrivinginplaceslc.org/}{Thriving in Place}" initiative, which aims to identify neighborhoods vulnerable to resident displacement and develop anti-displacement strategies. Focused on the Liberty Wells neighborhood, data was collected through resident surveys to assess changes in housing affordability and its impacts over time. Conducted statistical and empirical analyses to identify actionable policies that preserve, protect, and increase housing availability. Proposed strategies to help residents stay and thrive amid city growth, contributing to a broader understanding of gentrification and displacement in Salt Lake City.}
    \resumeItemListEnd

    \resumeProjectHeading
    {\textbf{\href{https://bhandaripukar.com.np/pt-portfolio/a-15-minute-salt-lake-city/}{15-Minute Salt Lake City: Pedestrian Accessibility Analysis, UT}} $|$ \emph{University of Utah}}{2022}
    \resumeItemListStart
        \resumeItem{Conducted an analysis of pedestrian accessibility in Salt Lake County as part of CMP 6720 – Land Use \& Transportation Planning coursework. The study evaluated how well different areas of the county align with the principles of a 15-minute city. Applied buffer analysis on geospatial data of public amenities (schools, health centers, groceries, parks, etc.), using a distance decay function to measure accessibility and the Analytical Hierarchical Process to weigh the importance of each amenity. The analysis found that high pedestrian accessibility is concentrated in Salt Lake City's downtown and Sugarhouse areas, with some outlying pockets of accessibility in places like Murray and Draper, though these are exceptions.}
    \resumeItemListEnd
    
    \resumeProjectHeading
    {\textbf{\href{https://bhandaripukar.com.np/pt-portfolio/greywater-recycling-at-west-village-redevelopment/}{Greywater Recycling at West Village Redevelopment at University of Utah}} $|$ \emph{University of Utah}}{2021}
    \resumeItemListStart
        \resumeItem{Conducted as part of the CMP 6610 - Urban Ecology coursework, developed a conceptual plan to enhance sustainability at the University of Utah through greywater recycling. Analyzed water usage data and reviewed literature on greywater production to identify strategies for reducing water usage and recycling greywater, particularly for landscaping. Proposed the integration of a greywater recycling facility within the West Village Student Housing redevelopment, alongside mixed recycling facilities for storing the recycled water. This cost-effective strategy aims to significantly reduce the university's water footprint and should be considered for future policy and project prioritization.}
    \resumeItemListEnd

    \resumeProjectHeading
    {\textbf{\href{https://bhandaripukar.com.np/pt-portfolio/suitability-of-bicycle-sharing-station-in-university-of-utah-campus/}{Suitability Analysis of Bicycle Sharing Station in University of Utah}} $|$ \emph{University of Utah}}{2021}
    \resumeItemListStart
        \resumeItem{Conducted as part of CMP 6700 - Walk \& Bike Planning/Design, this project aimed to support the University of Utah's sustainability initiatives by identifying optimal locations for bicycle sharing stations on campus. The study focused on selecting appropriate factors, variables, and criteria from literature reviews to inform the siting process. Utilizing the Analytical Hierarchy Process (AHP), raster images of the identified variables were overlaid to pinpoint potential station locations. The research recommended establishing micro-mobility stations at thirteen identified sites, while suggesting that the fleet size for bike-sharing should be determined through further analysis of user demand across the campus.}
    \resumeItemListEnd

    \resumeProjectHeading
    {\textbf{\href{https://bhandaripukar.com.np/pt-portfolio/salt-lake-city-profile-2010-2019/}{Salt Lake City Profile 2010-2019, UT}} $|$ \emph{University of Utah}}{2021}
    \resumeItemListStart
        \resumeItem{Developed a comprehensive city profile of Salt Lake City as part of the CMP 6010 – Community and Regional Analysis coursework. Conducted advanced analysis of demographic, economic, housing, and transportation data from primary sources such as the US Census Bureau to assess the city's historical and current conditions for the periods between 2010 and 2019. Projected future trends based on these analyses and complemented the findings with literature reviews and local knowledge. The resulting report offers a detailed overview of Salt Lake City's evolving needs and conditions, providing valuable insights for future city planning and development programs.}
    \resumeItemListEnd

    \resumeProjectHeading
    {\textbf{\href{https://storymaps.arcgis.com/stories/ed6db42d441e4b7b864d7969fac4ba3a}{Wasatch Hollow: Mapping the story of a changing neighborhood, UT}} $|$ \emph{University of Utah}}{2021}
    \resumeItemListStart
        \resumeItem{Conducted as part of CMP 6430 - Community Engagement in Planning, this project focused on documenting the changing conditions in the Wasatch Hollow neighborhood of Salt Lake City, particularly regarding housing unaffordability and displacement. The research involved reviewing historical records, analyzing secondary data, and conducting public interviews and surveys to gather firsthand experiences from residents. The findings were compiled into an ArcGIS StoryMap, which visually narrates the neighborhood's history, current conditions, and transformations in community assets while capturing recurring themes in the stories shared by residents.}
    \resumeItemListEnd

    \resumeProjectHeading
    {\textbf{\href{https://www.arcgis.com/apps/MapSeries/index.html?appid=a89311d3ac89482484c5cd421781f2ac}{Folsom Trail Extension along 400W in Salt Lake City, UT}} $|$ \emph{University of Utah}}{2021}
    \resumeItemListStart
        \resumeItem{As part of the CMP 6700 - Walk \& Bike Planning/Design coursework, developed a design for a 1,000 ft extension of the Folsom Trail along 400W, northwest of downtown Salt Lake City. The plan accommodated both bicycle and pedestrian users, with a focus on improving safety at two key intersections, particularly near West High School due to high student traffic. The design prioritized safety, accessibility, and enhanced multimodal experiences for trail users. The final submission included 2D and 3D visualizations (using Adobe Illustrator and SketchUp) and a 3D animation showcasing the proposed improvements.}
    \resumeItemListEnd

    \resumeProjectHeading
    {\textbf{\href{https://bhandaripukar.com.np/pt-portfolio/study-on-landcover-change-and-risk-analysis-of-west-kathmandu-valley/}{Landcover Change and Risk Analysis of West Kathmandu Valley}} $|$ \emph{Tribhuvan University}}{2021}
    \resumeItemListStart
        \resumeItem{A landcover change and risk analysis was conducted for West Kathmandu Valley as part of the "Geospatial Technologies for Urban Planning" course at Tribhuvan University. The project involved calculating descriptive statistics of spatio-temporal landcover changes using UNDP data and developing an integrated disaster risk map. An Analytical Hierarchical Process (AHP) was applied for hazard and socio-economic vulnerability analysis in GIS. The final analysis identified trends in landcover changes and highlighted high-risk areas, providing valuable insights for stakeholders to design spatially targeted policies for resilient and sustainable urban growth.}
    \resumeItemListEnd

    \resumeProjectHeading
    {\textbf{Masterplan of Padma Kanya Multiple Campus} $|$ \emph{IOE, Research Training and Consultancy Services}}{2020}
    \resumeItemListStart
        \resumeItem{Developed a comprehensive masterplan for Padma Kanya Multiple Campus, one of Nepal’s oldest women’s colleges, to address years of underdevelopment and prepare for future growth needs. The project involved reviewing the campus's strategic plan, conducting site investigations, and revising the space program to accommodate expansion. Architectural designs, digital models, and visualizations were created to communicate the redevelopment plan to stakeholders. Additionally, potential funding sources were identified to support future implementation. The final master plan provided a strategic framework for modernizing the campus and meeting future infrastructure demands.}
    \resumeItemListEnd

    \resumeProjectHeading
    {\textbf{Conservation and Development Work of Pashupatinath Temple Complex} $|$ \emph{Archetype Design Consult}}{2020}
    \resumeItemListStart
        \resumeItem{Pashupatinath Temple, a UNESCO World Heritage Site and one of the holiest Hindu temples, required a preservation and reconstruction plan following the damage from the 2015 Gorkha earthquake. Leading a team of architects, I conducted an assessment of nearly 30 monuments within the heritage zone and proposed conservation interventions aligned with best practices. Additionally, a master plan for redeveloping the Bagmati River ghat as a religious, cultural, and community space was developed in an effort to revitalize the entire monument area. The masterplan was successfully implemented by the Pashupati Area Development Trust, revitalizing the damaged monuments and enhancing the Guheswori Ghat area as a public space.}
    \resumeItemListEnd

    \resumeProjectHeading
    {\textbf{\href{https://bhandaripukar.com.np/sindhuligadhi-a-living-war-memorial/}{Conservation Masterplan of Singhuligadhi Area}} $|$ \emph{Archetype Design Consult}}{2019}
    \resumeItemListStart
        \resumeItem{Singhuligadhi, a historic hill fort central to Nepal's resistance against British forces in the 1800s, was slated for redevelopment as a war memorial. The task involved documenting the fort’s existing conditions, proposing conservation interventions, and designing a memorial park to attract historical tourism. The design team, led by me, documented the site, referenced historical records, and developed a strategy aligned with UNESCO and ICOMOS standards. The master plan was presented to the Department of Archaeology and local government, and the site is now redeveloped as a war memorial and museum following the plan.}
    \resumeItemListEnd

    \resumeProjectHeading
    {\textbf{\href{https://bhandaripukar.com.np/pt-portfolio/earthquake-memorial-at-barpak/}{Earthquake Memorial at Barpak, Gorkha}} $|$ \emph{Tribhuvan University (B.Arch Thesis)}}{2018}
    \resumeItemListStart
        \resumeItem{Three years after the 2015 Gorkha earthquake, the Barpak community, the earthquake's epicenter, was undergoing redevelopment. In response, the Government of Nepal initiated plans for an Earthquake Memorial to honor those affected, drive tourism, and revitalize the community. The task involved designing a memorial that not only commemorated lives lost but also symbolized resilience and served as a community refuge in future disasters. Extensive research on architectural elements of memory and experience, including form, materials, and lighting, was conducted to inform the design. The final design was submitted as the B.Arch thesis and earned the Best Architecture Thesis of the Year award from Asian Paints and the Department of Architecture.}
    \resumeItemListEnd

    \resumeProjectHeading
    {\textbf{\href{https://bhandaripukar.com.np/pt-portfolio/nuwakot-durbar-conservation/}{Conservation of Nuwakot Durbar Area}} $|$ \emph{Tribhuvan University}}{2017}
    \resumeItemListStart
        \resumeItem{The historic Nuwakot Durbar area, a key site in Nepal's unification and listed on UNESCO's tentative World Heritage Site list, was facing deterioration due to neglect and unplanned development. A conservation masterplan was developed to document existing monuments and create a framework for sustainable development that met contemporary community needs while preserving the heritage site’s historical significance. The project involved documenting key historic structures through architectural drawings in line with ICOMOS and UNESCO standards, drafting preservation bylaws, and creating visual materials to engage the community in the conservation process. The final documentation and bylaws were presented to the local community and submitted to the Nuwakot Area Development Authority and the Department of Archeology to support ongoing preservation efforts.}
    \resumeItemListEnd

    \resumeProjectHeading
    {\textbf{\href{https://bhandaripukar.com.np/pt-portfolio/barpak-resettlement-plan-ward-4/}{Barpak Resettlement Plan: Ward 4}} $|$ \emph{Tribhuvan University}}{2017}
    \resumeItemListStart
        \resumeItem{In response to the 2015 Gorkha earthquake, which demolished nearly 90\% of Barpak village, a conceptual masterplan was developed to aid in the reconstruction of a section of the village with a focus on resilience. The plan emphasized community connectivity, access to essential amenities, and the preservation of local architecture through four housing prototypes that integrated traditional construction methods with modern, earthquake-resistant techniques. The master plan and prototypes were presented to the local government to support their resettlement and rebuilding efforts.}
    \resumeItemListEnd

    \resumeProjectHeading
    {\textbf{\href{https://bhandaripukar.com.np/pt-portfolio/cafeteria-at-ioe-pulchowk-campus/}{Cafeteria at IOE, Pulchowk Campus}} $|$ \emph{Asian Paints National Design Competition}}{2017}
    \resumeItemListStart
        \resumeItem{For a national architectural design competition organized by Asian Paints, a conceptual cafeteria design was created for the IOE, Pulchowk Campus, near the Department of Architecture. The design featured an elevated structure utilizing wood and metal components, ensuring the preservation of existing trees by incorporating them into the layout. This design achieved runner-up status at the college level and qualified for the national competition.}
    \resumeItemListEnd

\resumeSubHeadingListEnd